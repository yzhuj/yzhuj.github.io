\documentclass[12pt]{report}
\usepackage[utf8]{inputenc}
\usepackage[english]{babel}
\usepackage[letterpaper, portrait, margin=1in]{geometry}
\usepackage{amsmath}
\numberwithin{equation}{section}
\usepackage{amssymb}
\usepackage{graphicx}
\usepackage{parskip}
\usepackage{xcolor}
\usepackage{physics}
\usepackage{empheq}
\usepackage{cancel}
\usepackage{hyperref}
\hypersetup{colorlinks = true, urlcolor = blue, linkcolor = red, citecolor = red}
\usepackage{enumerate}
\usepackage{tikz}
\usepackage{float}
\usepackage{tcolorbox}
\usepackage{booktabs}
\usepackage[bottom]{footmisc}

\newcommand{\unit}[1]{\hat{\boldsymbol{#1}}}
\newcommand{\bd}[1]{\boldsymbol{#1}}
\renewcommand{\b}[1]{\boldsymbol{#1}}
\renewcommand{\vec}[1]{\boldsymbol{#1}}

\definecolor{dmb}{HTML}{003366}

\begin{document}
	
	\title{\textbf{Quantum Mechanics: A Brief Overview\newline\\\normalsize From \textit{Quantum Mechanics} by David J. Griffiths \\and \textit{Quantum Mechanics} by B.H. Bransden \& C.J. Joachain}}
	\author{\textbf{Yi J. Zhu}}
	\date{August 24, 2020\\\vspace{1em}\small Updated \today}
	\maketitle
	\thispagestyle{empty}
	
	\tableofcontents
	
	\chapter{Formalism}

	Before discussing the formalism of quantum mechanics, we shall provide a brief review of linear algebra. 
	
	A vector is an N-tuple. A vector space $ \mathbb{F}^N $ is a set of vectors over the field $ \mathbb{F} $. Vectors in a vector space are commutative in addition, associative in addition, and distributive with respect to a scalar. Furthermore, all vector fields by definition contain an additive identity, additive inverse, and multiplicative identity. Vectors in quantum mechanics are represented by kets,
	\begin{equation}
			\ket{\alpha} \doteq \begin{pmatrix}a_1\\a_2\\\vdots\\a_N\end{pmatrix}
	\end{equation}
	
	Furthermore, the inner product is a function on a vector space that takes a pair of vectors to a scalar. The inner product of vector kets $ \ket{\alpha} $ and $ \ket{\beta} $ are denoted $ \braket{\alpha}{\beta} $ and the inner product has properties, 
	\begin{enumerate}
		\item positive: $ \braket{\alpha}{\alpha} \geq 0$
		\item definite: $ \braket{\alpha}{\alpha} = 0$ if and only if $ \ket{\alpha} = 0 $ 
	\end{enumerate}

	\begin{equation}
			\sigma_i = \pm 1
	\end{equation}
	
	\bibliographystyle{ieeetr}
	\bibliography{ref.bib}
	\nocite{*}
	
	\appendix
	\chapter{Field and Source Points}

\end{document}