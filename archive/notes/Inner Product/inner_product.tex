\documentclass[12pt]{article}
\usepackage[utf8]{inputenc}
\usepackage[english]{babel}
\usepackage{amsmath}
\numberwithin{equation}{section}
\usepackage{amssymb}
\usepackage{graphicx}
\usepackage{wrapfig}
\usepackage{parskip}
\usepackage{xcolor}
\usepackage{physics}
\usepackage{empheq}
\usepackage{cancel}
\usepackage{hyperref}
\hypersetup{colorlinks = true, urlcolor = blue, linkcolor = red, citecolor = red}
\usepackage{enumerate}
\usepackage{tikz}
\usepackage{float}
\usepackage{tcolorbox}
\usepackage{booktabs}

\begin{document}
	\title{The Inner Product}
	\author{Yi J Zhu}
	\date{January 16, 2022}
	\maketitle

	\vspace{1cm}
	
	\textbf{\textit{Definition (Hermitian inner product)}}: Let $ V $ be a complex vector space. A Hermitian inner product on $ V $ is a function,
	\begin{equation}
			\langle \cdot,\cdot\rangle: V\times V \to \mathbb{C}
	\end{equation}

	which is,
	\begin{itemize}
		\item \textbf{conjugate-symmetric}
		\begin{equation}
				\langle u, v\rangle = \langle v,u\rangle^*
		\end{equation}
		\item \textbf{sesquilinear}, linear in the second term (physics convention). For arbitrary scalar $\lambda$,
		\begin{align}
			&\langle u,\lambda v\rangle = \lambda \langle u, v\rangle\\
			&\langle \lambda u, v\rangle = \lambda^* \langle u, v\rangle\\
			&\langle u, v_1 + v_2 \rangle = \langle  u, v_1\rangle + \langle  u, v_2\rangle
		\end{align}
		\item \textbf{Positive-definite},
		\begin{equation}
				\langle \lambda u, u\rangle \geq 0,\quad\text{equality iff }u=0
		\end{equation}
		
	\end{itemize}

	\textbf{\textit{Corollary:}} An orthonormal set of vectors $ \{e_i\} $ in $ V $ is linearly independent. \textit{Note: the converse is not true.}
	\begin{equation}
		\langle e_i, e_j\rangle = \delta_{ij}\quad \text{(orthonormality)}
	\end{equation}


\newpage

\textbf{Orthonormal basis}: Suppose $ v\in V $ and $ \{e_i\} $ is an orthonormal basis,
\begin{equation}
	v = \sum_i v_i e_i
\end{equation}

We will find that an orthonormal basis has useful consequences, so do all vector spaces have an orthonormal basis? In a finite-dimension space the answer is easy. Yes: all vector spaces have a basis, use the Gram–Schmidt algorithm to determine an orthonormal basis.

\textbf{Projection:} What is the component $ v_j $ (linear-expansion coefficient) of $ v $ on the orthonormal basis vector $ e_j $? In Einstein notation,
\begin{equation}
		\langle e_j, v\rangle = \langle e_j, v_ie_i\rangle = v_i\langle e_j, e_i\rangle = v_j
\end{equation}

\textbf{Dot product:} Suppose $ u,v\in V $ ,
\begin{align}
	\langle u,v\rangle &= \langle u_ie_i,v_je_j\rangle \\
	&= u_i^* v_j\langle e_i, e_j \rangle\\
	&= u_i^* v_i
\end{align}

Therefore, the inner product of two vectors has particular significance in terms of their expansion coefficients in an \textit{arbitrary} orthonormal basis. 

What if $ v=u $; i.e. $ \langle u,u\rangle $? This forms the notion of a metric.

	\textbf{\textit{Definition (norm)}}: Let $ V $ be a complex vector space. The norm of a vector $ v\in V $ is a function,
	\begin{equation}
			||\cdot ||: V\to \mathbb{R}
	\end{equation}
	defined as,
	\begin{equation}
			||v|| = \sqrt{\langle v, v\rangle}
	\end{equation}


\end{document}